\chapter{Modelos Te\'{o}ricos}
La din\'{a}mica de una biomol\'{e}cula se determina por las ecuaciones de movimiento de los \'{a}tomos que la constituyen. Usualmente en una biomol\'{e}cula el n\'{u}mero de mon\'{o}meros es mayor a 20, que al multiplicarlo por el n\'{u}mero de \'{a}tomos en cada mon\'{o}mero incrementa considerablemente el n\'{u}mero de ecuaciones de movimiento a resolver, entonces se hace necesario realizar \textit{din\'{a}mica molecular} (Molecular Dynamics por sus siglas en ingl\'{e}s MD) en la cual se estudia mediante simulaciones computacionales el movimiento de los \'{a}tomos de acuerdo a las interacciones que presenten.\\

Las ecuaciones de movimiento se pueden conocer al aplicar los formalismos lagrangiano o hamiltoniano en los cuales es necesario conocer los potenciales con los que interact\'{u}an los \'{a}tomos puede ser descrita mediante los m\'{e}todos de la din\'{a}mica molecular o los an\'{a}lisis de modos normales (Normal Mode Analysis por sus siglas en ingl\'{e}s NMA). Los potenciales usados en \cite{G1,AG09b}.

\section{Subt\'{\i}tulos nivel 2}
Toda divisi\'{o}n o cap\'{\i}tulo, a su vez, puede subdividirse en otros niveles y s\'{o}lo se enumera hasta el tercer nivel. Los t\'{\i}tulos de segundo nivel se escriben con min\'{u}scula al margen izquierdo y sin punto final, est\'{a}n separados del texto o contenido por un interlineado posterior de 10 puntos y anterior de 20 puntos (tal y como se presenta en la plantilla).\\

\subsection{Subt\'{\i}tulos nivel 3}
De la cuarta subdivisi\'{o}n en adelante, cada nueva divisi\'{o}n o \'{\i}tem puede ser se\~{n}alada con vi\~{n}etas, conservando el mismo estilo de \'{e}sta, a lo largo de todo el documento.\\

Las subdivisiones, las vi\~{n}etas y sus textos acompa\~{n}antes deben presentarse sin sangr\'{\i}a y justificados.\\

\begin{itemize}
\item En caso que sea necesario utilizar vi\~{n}etas, use este formato (vi\~{n}etas cuadradas).
\end{itemize}